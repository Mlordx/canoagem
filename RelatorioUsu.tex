\documentclass[11pt]{article}
\usepackage[brazilian]{babel}
\usepackage[utf8]{inputenc} %Deixa eu colocar letras com ascentos
\usepackage[T1]{fontenc}
\usepackage{amsmath}
\usepackage{color}
\usepackage{indentfirst}

\title{Exercício Programa 3  - Guia do Usuário}

\begin{document}

\maketitle

\section{Modo de compilaç\~ ao}

Para gerar o executável do programa, basta executar o seguinte comando na pasta onde está localizado os arquivos do programa.

\begin{itemize}

\item[>]make

\end{itemize}

Com isso, será gerado o executável chamado ep3.

\section{Modo de execução}

O programa tem os seguintes parametros que podem ser inseridos via linha de comando (substituir o que estiver entre chaves adequadamente):

\begin{itemize}

\item[+] \textbf{-L[{\color{red}N}]:} Argumento dizendo o número de linhas que o rio terá. Caso esse argumento não seja passado, um valor default será usado.

\item[+] \textbf{-C[{\color{red}N}]:} Argumento dizendo o número de colunas que o rio terá. Caso esse argumento não seja passado, um valor default será usado.

\item[+] \textbf{-s[{\color{red}seed}]:} Argumento que diz a seed dos números aleatórios que serão usados no programa. Caso esse argumento não seja passado, um valor default será usado.

\item[+] \textbf{-t[{\color{red}tam}]:} Argumento que diz o tamanho mínimo que o rio deverá ter. Caso nenhum valor seja passado, será usado um valor padrão.

\item[+] \textbf{-F[{\color{red}flux}]:} Argumento que diz o fluxo que o rio deverá ter. Caso nenhum valor seja passado, será usado um valor padrão.

\item[+] \textbf{-T[{\color{red}X}]:} Argumento que diz para o programa executar em modo de teste. Esse comando pode ser repetido quantas vezes necessárias, sendo X um número inteiro variando de 1 a 4. De acordo com o número, será execuigualaFluxotado um determinado teste:
	\begin{enumerate}
		\item Testa um rio com um número grande de linhas.
		\item Testa um rio com fluxo 0.
		\item Testa um rio de tamanho mínimo 0.
		\item Testa as variações da margem do rio.		
	
	\end{enumerate}
	
\item[+] \textbf{-I[{\color{red}it}]:} Define quantas iterações, ou quantas vezes o rio será atualizado em cada teste. Esse argumento não tem efeito quando o modo de execução não é o de teste.

\end{itemize}

Exemplos de execução do programa(em bash):
\begin{itemize}
	\item[>] ./ep3 -L30 -C100 -s666 -F11.5
	\item[>] ./ep3\footnote{Irá executar um rio com valores padrões}
	\item[>] ./ep3 -T1 -T4 -I10000
	\item[>] ./ep3 -T1 -s11 -I500



\end{itemize}

\end{document}